The following list outlines the FUN3D inputs along with their default values available through the AIM interface. One will note most of the FUN3D parameters have a NULL value as their default. This is done since a parameter in the FUN3D input deck (fun3d.\+nml) is only changed if the value has been changed in CAPS (i.\+e. set to something other than NULL).


\begin{DoxyItemize}
\item {\bfseries{ Proj\+\_\+\+Name = \char`\"{}fun3d\+\_\+\+CAPS\char`\"{}}} ~\newline
 This corresponds to the project\+\_\+rootname variable in the \&project namelist of fun3d.\+nml.
\item {\bfseries{ Mach = NULL }} ~\newline
 This corresponds to the mach\+\_\+number variable in the \&reference\+\_\+physical\+\_\+properties namelist of fun3d.\+nml.
\item {\bfseries{ Re = NULL }} ~\newline
 This corresponds to the reynolds\+\_\+number variable in the \&reference\+\_\+physical\+\_\+properties namelist of fun3d.\+nml.
\item {\bfseries{ Temperature = NULL }} ~\newline
 This corresponds to the temperature variable in the \&reference\+\_\+physical\+\_\+properties namelist of fun3d.\+nml. Note if no temperature units are set, units of Kelvin are assumed (see \mbox{\hyperlink{aimUnitsFUN3D}{AIM Units}})
\item {\bfseries{ Viscous = NULL }} ~\newline
 This corresponds to the viscous\+\_\+terms variable in the \&governing\+\_\+equation namelist of fun3d.\+nml.
\item {\bfseries{ Equation\+\_\+\+Type = NULL }} ~\newline
 This corresponds to the eqn\+\_\+type variable in the \&governing\+\_\+equation namelist of fun3d.\+nml.
\item {\bfseries{ Alpha = NULL }} ~\newline
 This corresponds to the angle\+\_\+of\+\_\+attack variable in the \&reference\+\_\+physical\+\_\+properties namelist of fun3d.\+nml \mbox{[}degree\mbox{]}.
\item {\bfseries{ Beta = NULL }} ~\newline
 This corresponds to the angle\+\_\+of\+\_\+yaw variable in the \&reference\+\_\+physical\+\_\+properties namelist of fun3d.\+nml \mbox{[}degree\mbox{]}.
\item {\bfseries{ Overwrite\+\_\+\+NML = NULL}} ~\newline

\begin{DoxyItemize}
\item If Python is NOT linked with the FUN3D AIM at compile time or Use\+\_\+\+Python\+\_\+\+NML is set to False this flag gives the AIM permission to overwrite fun3d.\+nml if present. The namelist produced will solely consist of input variables present and set in the AIM.
\item If Python IS linked with the FUN3D AIM at compile time and Use\+\_\+\+Python\+\_\+\+NML is set to True the namelist file will be overwritten, as opposed to being appended.
\end{DoxyItemize}
\item {\bfseries{Mesh\+\_\+\+Format = \char`\"{}\+AFLR3\char`\"{}}} ~\newline
 Mesh output format. By default, an AFLR3 mesh will be used.
\item {\bfseries{Mesh\+\_\+\+ASCII\+\_\+\+Flag = True}} ~\newline
 Output mesh in ASCII format, otherwise write a binary file if applicable.
\item {\bfseries{Num\+\_\+\+Iter = NULL}} ~\newline
 This corresponds to the steps variable in the \&code\+\_\+run\+\_\+control namelist of fun3d.\+nml.
\item {\bfseries{CFL\+\_\+\+Schedule = NULL}} ~\newline
 This corresponds to the schedule\+\_\+cfl variable in the \&nonlinear\+\_\+solver\+\_\+parameters namelist of fun3d.\+nml.
\item {\bfseries{CFL\+\_\+\+Schedule\+\_\+\+Inter = NULL}} ~\newline
 This corresponds to the schedule\+\_\+iteration variable in the \&nonlinear\+\_\+solver\+\_\+parameters namelist of fun3d.\+nml.
\item {\bfseries{Restart\+\_\+\+Read = NULL}} ~\newline
 This corresponds to the restart\+\_\+read variable in the \&code\+\_\+run\+\_\+control namelist of fun3d.\+nml.
\item {\bfseries{Boundary\+\_\+\+Condition = NULL }} ~\newline
 See \mbox{\hyperlink{cfdBoundaryConditions}{CFD Boundary Conditions}} for additional details.
\item {\bfseries{Use\+\_\+\+Python\+\_\+\+NML = False }} ~\newline
 By default, even if Python has been linked to the FUN3D AIM it is not used unless the this value is set to True.
\item {\bfseries{Pressure\+\_\+\+Scale\+\_\+\+Factor = 1.\+0}} ~\newline
 Value to scale Cp data when transferring data. Data is scaled based on Pressure = Pressure\+\_\+\+Scale\+\_\+\+Factor$\ast$\+Cp + Pressure\+\_\+\+Scale\+\_\+\+Offset.
\item {\bfseries{Pressure\+\_\+\+Scale\+\_\+\+Offset = 0.\+0}} ~\newline
 Value to offset Cp data when transferring data. Data is scaled based on Pressure = Pressure\+\_\+\+Scale\+\_\+\+Factor$\ast$\+Cp + Pressure\+\_\+\+Scale\+\_\+\+Offset.
\item {\bfseries{Temperature\+\_\+\+Scale\+\_\+\+Factor = 1.\+0}} ~\newline
 Value to scale Temperature data when transferring data.
\item {\bfseries{Non\+Inertial\+\_\+\+Rotation\+\_\+\+Rate = NULL \mbox{[}0.\+0, 0.\+0, 0.\+0\mbox{]}}} ~\newline
 Array values correspond to the rotation\+\_\+rate\+\_\+x, rotation\+\_\+rate\+\_\+y, rotation\+\_\+rate\+\_\+z variables, respectively, in the \&noninertial\+\_\+reference\+\_\+frame namelist of fun3d.\+nml.
\item {\bfseries{Non\+Inertial\+\_\+\+Rotation\+\_\+\+Center = NULL, \mbox{[}0.\+0, 0.\+0, 0.\+0\mbox{]}}} ~\newline
 Array values correspond to the rotation\+\_\+center\+\_\+x, rotation\+\_\+center\+\_\+y, rotation\+\_\+center\+\_\+z variables, respectively, in the \&noninertial\+\_\+reference\+\_\+frame namelist of fun3d.\+nml.
\item {\bfseries{Two\+\_\+\+Dimensional = False}} ~\newline
 Run FUN3D in 2D mode. If set to True, the body must be a single \char`\"{}sheet\char`\"{} body in the x-\/z plane (a rudimentary node swapping routine is attempted if not in the x-\/z plane). A 3D mesh will be written out, where the body is extruded a length of 1 in the y-\/direction.
\item {\bfseries{Modal\+\_\+\+Aeroelastic = NULL }} ~\newline
 See \mbox{\hyperlink{cfdModalAeroelastic}{CFD Modal Aeroelastic}} for additional details.
\item {\bfseries{Modal\+\_\+\+Ref\+\_\+\+Velocity = NULL }} ~\newline
 The freestream velocity in structural dynamics equation units; used for scaling during modal aeroelastic simulations. This corresponds to the uinf variable in the \&aeroelastic\+\_\+modal\+\_\+data namelist of movingbody.\+input.
\item {\bfseries{Modal\+\_\+\+Ref\+\_\+\+Length = 1.\+0 }} ~\newline
 The scaling factor between CFD and the structural dynamics equation units; used for scaling during modal aeroelastic simulations. This corresponds to the grefl variable in the \&aeroelastic\+\_\+modal\+\_\+data namelist of movingbody.\+input.
\item {\bfseries{Modal\+\_\+\+Ref\+\_\+\+Dynamic\+\_\+\+Pressure = NULL }} ~\newline
 The freestream dynamic pressure in structural dynamics equation units; used for scaling during modal aeroelastic simulations. This corresponds to the qinf variable in the \&aeroelastic\+\_\+modal\+\_\+data namelist of movingbody.\+input.
\item {\bfseries{Time\+\_\+\+Accuracy = NULL }} ~\newline
 Defines the temporal scheme to use. This corresponds to the time\+\_\+accuracy variable in the \&nonlinear\+\_\+solver\+\_\+parameters namelist of fun3d.\+nml.
\item {\bfseries{Time\+\_\+\+Step = NULL }} ~\newline
 Non-\/dimensional time step during time accurate simulations. This corresponds to the time\+\_\+step\+\_\+nondim variable in the \&nonlinear\+\_\+solver\+\_\+parameters namelist of fun3d.\+nml.
\item {\bfseries{Num\+\_\+\+Subiter = NULL }} ~\newline
 Number of subiterations used during a time step in a time accurate simulations. This corresponds to the subiterations variable in the \&nonlinear\+\_\+solver\+\_\+parameters namelist of fun3d.\+nml.
\item {\bfseries{Temporal\+\_\+\+Error = NULL }} ~\newline
 This sets the tolerance for which subiterations are stopped during time accurate simulations. This corresponds to the temporal\+\_\+err\+\_\+floor variable in the \&nonlinear\+\_\+solver\+\_\+parameters namelist of fun3d.\+nml.
\item {\bfseries{Reference\+\_\+\+Area = NULL }} ~\newline
 This sets the reference area for used in force and moment calculations. This corresponds to the area\+\_\+reference variable in the \&force\+\_\+moment\+\_\+integ\+\_\+properties namelist of fun3d.\+nml. Alternatively, the geometry (body) attribute \char`\"{}caps\+Reference\+Area\char`\"{} maybe used to specify this variable (note\+: values set through the AIM input will supersede the attribution value).
\item {\bfseries{Moment\+\_\+\+Length = NULL, \mbox{[}0.\+0, 0.\+0\mbox{]}}} ~\newline
 Array values correspond to the x\+\_\+moment\+\_\+length and y\+\_\+moment\+\_\+length variables, respectively, in the \&force\+\_\+moment\+\_\+integ\+\_\+properties namelist of fun3d.\+nml. Alternatively, the geometry (body) attributes \char`\"{}caps\+Reference\+Chord\char`\"{} and \char`\"{}caps\+Reference\+Span\char`\"{} may be used to specify the x-\/ and y-\/ moment lengths, respectively (note\+: values set through the AIM input will supersede the attribution values).
\item {\bfseries{Moment\+\_\+\+Center = NULL, \mbox{[}0.\+0, 0.\+0, 0.\+0\mbox{]}}} ~\newline
 Array values correspond to the x\+\_\+moment\+\_\+center, y\+\_\+moment\+\_\+center, and z\+\_\+moment\+\_\+center variables, respectively, in the \&force\+\_\+moment\+\_\+integ\+\_\+properties namelist of fun3d.\+nml. Alternatively, the geometry (body) attributes \char`\"{}caps\+Reference\+X\char`\"{}, \char`\"{}caps\+Reference\+Y\char`\"{}, and \char`\"{}caps\+Reference\+Z\char`\"{} may be used to specify the x-\/, y-\/, and z-\/ moment centers, respectively (note\+: values set through the AIM input will supersede the attribution values).
\item {\bfseries{FUN3\+D\+\_\+\+Version = 13.\+1 }} ~\newline
 FUN3D version to generate specific configuration file for; currently only has influence over rubber.\+data (sensitivity file) and aeroelastic modal data namelist in moving\+\_\+body.\+input .
\item {\bfseries{Motion\+\_\+\+Driver = NULL }} ~\newline
 Triggers writing moving\+\_\+body.\+input with specified \textquotesingle{}motion\+\_\+driver\textquotesingle{} string.~\newline
 Mesh\+\_\+\+Movement input must also be set.
\item {\bfseries{Mesh\+\_\+\+Movement = NULL }} ~\newline
 Triggers writing moving\+\_\+body.\+input with specified \textquotesingle{}mesh\+\_\+movement\textquotesingle{} string.~\newline
 Motion\+\_\+\+Driver input must also be set.
\item {\bfseries{ Design\+\_\+\+Variable = NULL}} ~\newline
 List of Analysis\+In and/or Geometry\+In variable names used to compute sensitivities of Design\+\_\+\+Functional for optimization, see \mbox{\hyperlink{cfdDesignVariable}{CFD Design Variable}} for additional details.
\item {\bfseries{ Design\+\_\+\+Functional = NULL}} ~\newline
 The design functional tuple is used to input functional information for optimization, see \mbox{\hyperlink{cfdDesignFunctional}{CFD Functional}} for additional details. Using this requires Design\+\_\+\+Sens\+File = False.
\item {\bfseries{ Design\+\_\+\+Sens\+File = False}} ~\newline
 Read $<$\+Proj\+\_\+\+Name$>$.sens file to compute functional sensitivities w.\+r.\+t Design\+\_\+\+Variable. Using this requires Design\+\_\+\+Functional = NULL.
\item {\bfseries{ Design\+\_\+\+Sensitivity = False}} ~\newline
 If True and Design\+\_\+\+Functional is set, create geometric sensitivities Fun3D input files needed to compute Design\+\_\+\+Functional sensitivities w.\+r.\+t Design\+\_\+\+Variable. If True and Design\+\_\+\+Sens\+File = True, read functional sensitivities from $<$\+Proj\+\_\+\+Name$>$.sens and compute sensitivities w.\+r.\+t Design\+\_\+\+Variable. The value of the design functionals become available as Dynamic Output Value Objects using the \char`\"{}name\char`\"{} of the functionals.
\item {\bfseries{ Mesh\+\_\+\+Morph = False}} ~\newline
 Project previous surface mesh onto new geometry and write out a \textquotesingle{}Proj\+\_\+\+Name\textquotesingle{}\+\_\+body\#.dat file.
\item {\bfseries{ Mesh\+\_\+\+Morph\+\_\+\+Combine = True}} ~\newline
 When using Mesh\+\_\+\+Morph, setting Mesh\+\_\+\+Morph\+\_\+\+Combine = True will write out a single body,\textquotesingle{}Proj\+\_\+\+Name\textquotesingle{}\+\_\+body1.\+dat, file containing all surface meshes. If Mesh\+\_\+\+Morph\+\_\+\+Combine = False, a body file will be written out for each individual surface mesh.
\item {\bfseries{Mesh = NULL}} ~\newline
 An Area\+\_\+\+Mesh or Volume\+\_\+\+Mesh link for 2D and 3D calculations respectively. 
\end{DoxyItemize}